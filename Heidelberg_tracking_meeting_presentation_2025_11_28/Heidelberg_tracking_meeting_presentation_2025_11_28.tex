%%%%%%%%%%%%%%%%%%%%%%%%%%%%%%%%%%%%%%%%%%%%%%%%%%%%%%%%%%%%%%%%%%%%%%
% Overleaf (WriteLaTeX) Example: Molecular Chemistry Presentation
%
% Source: http://www.overleaf.com
%
% In these slides we show how Overleaf can be used with standard
% chemistry packages to easily create professional presentations.
%
% Feel free to distribute this example, but please keep the referral
% to overleaf.com
%
%%%%%%%%%%%%%%%%%%%%%%%%%%%%%%%%%%%%%%%%%%%%%%%%%%%%%%%%%%%%%%%%%%%%%%

\documentclass[xcolor={dvipsnames}]{beamer}

\mode<presentation>
{
  \usetheme{Madrid}       % or try default, Darmstadt, Warsaw, ...
  \usecolortheme{default} % or try albatross, beaver, crane, ...
  \usefonttheme{default}    % or try default, structurebold, ...
  \setbeamertemplate{navigation symbols}{}
  \setbeamertemplate{caption}[numbered]
}

\usepackage[english]{babel}
\usepackage[utf8x]{inputenc}
\usepackage{graphicx}
\usepackage{hyperref}
  \hypersetup{colorlinks=true}
  \hypersetup{urlcolor=blue}
  \hypersetup{linkcolor = .}
\usepackage{xcolor}
\usepackage{siunitx}
  \sisetup{separate-uncertainty = true}
\DeclareSIUnit\barn{b}
\usepackage{physics}
\usepackage[font=small,labelfont=bf]{caption}
\usepackage{subcaption}
\usepackage[en-GB]{datetime2}
\usepackage{overpic}
\usepackage{feynmp}
\DeclareGraphicsRule{*}{mps}{*}{}
\usepackage{scalerel}
\newcommand{\mylbrace}[2]{\vspace{#2pt}\hspace{6pt}\scaleleftright[\dimexpr5pt+#1\dimexpr0.06pt]{\lbrace}{\rule[\dimexpr2pt-#1\dimexpr0.5pt]{-4pt}{#1pt}}{.}}
\newcommand{\myrbrace}[2]{\vspace{#2pt}\scaleleftright[\dimexpr5pt+#1\dimexpr0.06pt]{.}{\rule[\dimexpr2pt-#1\dimexpr0.5pt]{-4pt}{#1pt}}{\rbrace}\hspace{6pt}}

% Trim in percent
\usepackage{adjustbox}

% No "Figure" prefix
\setbeamertemplate{caption}{\raggedright\insertcaption\par}

% Nice decay amplitude diagrams
\usepackage{amsmath,amssymb,tikz-cd}

% Strike out text
\usepackage[normalem]{ulem}

% For figures with text overlay
\usepackage{overpic}

% Arrows
\usepackage{tikz}
\newcommand{\tikzmark}[1]{\tikz[remember picture] \node[coordinate] (#1) {#1};}

% Colourbox with line breaks
\newcommand{\cbox}[2][lime!20]{%
  \colorbox{#1}{\parbox{\dimexpr\linewidth-2\fboxsep}{\strut #2\strut}}%
}

% Vector arrows
\usepackage[pdftex]{pict2e}

% Checkmark symbol
\def\checkmark{\tikz\fill[scale=0.4](0,.35) -- (.25,0) -- (1,.7) -- (.25,.15) -- cycle;}

% Here's where the presentation starts, with the info for the title slide
\title[Heidelberg tracking meeting]{Impact of wrong matching on tracking efficiencies}

\author[Martin Tat]{Martin Tat}
\institute[Heidelberg]{Heidelberg University}
\date{28th November 2025}

\titlegraphic{\includegraphics[height = 2.3cm]{lhcb.jpg}\hspace{1.0cm}~%
              \includegraphics[height = 2.3cm]{HeidelbergLogo.pdf}}

\begin{document}

\begin{frame}
  \titlepage
\end{frame}

% These three lines create an automatically generated table of contents.
%\begin{frame}{Outline}
%  \tableofcontents
%\end{frame}

\begin{frame}{Introduction}
  \vspace{0.0cm}
  {\Large Recap from previous presentations:}
  \vspace{0.2cm}
  \begin{itemize}
    \setlength\itemsep{0.8em}
    \item{Tracking efficiencies with VeloMuon, Downstream, MuonUT methods}
    \begin{itemize}
      \item{Combined: VeloMuon $\times$ Downstream}
      \item{Cross check: MuonUT}
    \end{itemize}
    \item{A long-standing issue is a discrepancy between the two methods}
    \begin{itemize}
      \item{Efficiencies from MuonUT method are consistently higher than those from the combined method}
      \item{Data/MC ratios also show discrepancies}
    \end{itemize}
  \end{itemize}
\end{frame}

\begin{frame}{Block 5 tracking efficiencies}
  \vspace{0.0cm}
  \begin{figure}[htb]
    \centering
    \begin{subfigure}{0.5\textwidth}
      \centering
      \includegraphics[width=1.0\textwidth]{Plots/trackEff_MC_Sim10d_2024_Block5_Combined_P-ETA.pdf}
    \end{subfigure}%
    \begin{subfigure}{0.5\textwidth}
      \centering
      \includegraphics[width=1.0\textwidth]{Plots/trackEff_MC_Sim10d_2024_Block5_MuonUT_P-ETA.pdf}
    \end{subfigure}
  \end{figure}
  \begin{itemize}
    \item{Effect is worse in data}
    \item{Does not seem to cancel in data/MC ratio}
  \end{itemize}
\end{frame}

\begin{frame}{Checks performed}
  \vspace{0.0cm}
  {\large Current work: Try to understand why this discrepancy exists in MC, and whether or not it could explain some of the discrepancy in data}
  \vspace{0.2cm}
  \begin{itemize}
    \setlength\itemsep{0.8em}
    \item{Different kinematics?}
    \begin{itemize}
      \item{Finer bins still show discrepancies}
    \end{itemize}
    \item{MuonUT charge asymmetry bug?}
    \begin{itemize}
      \item{Fits split by charge consistent with charge integrated results}
    \end{itemize}
    \item{Bad fit model or counting bias?}
    \begin{itemize}
      \item{Truth matched MC still shows discrepancies}
    \end{itemize}
    \item{Problem must be somewhere further upstream...}
    \begin{itemize}
      \item{... so I looked into how the matching is performed}
    \end{itemize}
  \end{itemize}
\end{frame}

\begin{frame}{Checks performed}
  \vspace{0.0cm}
  {\large Matching is performed by compared hits in each subdetector}
  \vspace{0.2cm}
  \begin{center}
    \begin{tabular}{ccccc}
      \hline
      Method     & Velo  & UT & SciFi & Muon \\
      \hline
      VeloMuon   & $0.4$ & -- & --    & $0.4$ \\
      Downstream & --    & -- & $0.4$ & $0.4$ \\
      MuonUT     & --    & -- & --    & $0.4$ \\
      \hline
    \end{tabular}
  \end{center}
  \begin{itemize}
    \setlength\itemsep{0.8em}
    \item{Probe track is perhaps matched to a random long track}
    \item{Especially for MuonUT, which only has 4 hits in the Muon system, this effect should be checked}
    \item{Strategy: Copy Rowina's AP on MC and add long track truth info}
    \begin{itemize}
      \item{Compare \texttt{TRUEKEY} between probe track and long track}
    \end{itemize}
    \item{Warning: Results from local testjob have limited statistics}
    \begin{itemize}
      \item{AP was submitted yesterday and still running}
    \end{itemize}
  \end{itemize}
\end{frame}

\begin{frame}{Tracking efficiencies on truth matched yields}
  \vspace{0.0cm}
  {\large Calculate tracking efficiencies on truth matched yields (no fitting)}
  \vspace{0.2cm}
  \begin{center}
    \begin{tabular}{ccccc}
      \hline
      Method     & Charge & $N_{\rm pass}$ & $N_{\rm fail}$ & $\epsilon$ \\
      \hline
      VeloMuon   & $\mu^+$ & $594 \pm 24$ & $15 \pm 4$    & $0.975 \pm 0.006$ \\
      Downstream & $\mu^+$ & $400 \pm 20$ & $6.0 \pm 2.4$ & $0.985 \pm 0.006$ \\
      Combined   & $\mu^+$ & --           & --            & $0.961 \pm 0.009$ \\
      MuonUT     & $\mu^+$ & $97 \pm 10$  & $1.0 \pm 1.0$ & $0.990 \pm 0.010$ \\
      \hline
      VeloMuon   & $\mu^-$ & $625 \pm 25$ & $21 \pm 5$    & $0.967 \pm 0.007$ \\
      Downstream & $\mu^-$ & $428 \pm 21$ & $6.0 \pm 2.4$ & $0.986 \pm 0.006$ \\
      Combined   & $\mu^-$ & --           & --            & $0.954 \pm 0.009$ \\
      MuonUT     & $\mu^-$ & $77 \pm 9$   & $2.0 \pm 1.4$ & $0.975 \pm 0.018$ \\
      \hline
    \end{tabular}
  \end{center}
\end{frame}

\begin{frame}{Fraction of wrongly matched tracks}
  \vspace{0.0cm}
  {\large How often is the matched long track wrong?}
  \vspace{0.2cm}
  \begin{center}
    \begin{tabular}{ccccc}
      \hline
      Method     & Charge & $N_{\rm pass}$ & $N_{\rm pass}^{\rm wrong}$ & Correction \\
      \hline
      VeloMuon   & $\mu^+$ & $594 \pm 24$ & $0$     & $1$ \\
      Downstream & $\mu^+$ & $400 \pm 20$ & $11.0 \pm 3.3$& $0.972 \pm 0.008$ \\
      MuonUT     & $\mu^+$ & $97 \pm 10$  & $4.0 \pm 2.0$ & $0.959 \pm 0.020$ \\
      \hline
      VeloMuon   & $\mu^-$ & $625 \pm 25$ & $0$     & $1$ \\
      Downstream & $\mu^-$ & $428 \pm 21$ & $5.0 \pm 2.2$ & $0.988 \pm 0.005$ \\
      MuonUT     & $\mu^-$ & $77 \pm 9$   & $3.0 \pm 1.7$ & $0.961 \pm 0.022$ \\
      \hline
    \end{tabular}
  \end{center}
  \vspace{0.5cm}
  {\large Correction factor: $(N_{\rm pass} - N_{\rm pass}^{\rm wrong}) / N_{\rm pass}$}
\end{frame}

\begin{frame}{Corrected tracking efficiencies on truth matched yields}
  \vspace{0.0cm}
  {\large Multiply tracking efficiencies with correction factors}
  \vspace{0.2cm}
  \begin{center}
    \begin{tabular}{ccc}
      \hline
      Method     & Charge & $\epsilon$ \\
      \hline
      VeloMuon   & $\mu^+$ & $0.975 \pm 0.006$ \\
      Downstream & $\mu^+$ & $0.958 \pm 0.010$ \\
      Combined   & $\mu^+$ & $0.934 \pm 0.011$ \\
      MuonUT     & $\mu^+$ & $0.949 \pm 0.022$ \\
      \hline
      VeloMuon   & $\mu^-$ & $0.967 \pm 0.007$ \\
      Downstream & $\mu^-$ & $0.975 \pm 0.008$ \\
      Combined   & $\mu^-$ & $0.943 \pm 0.010$ \\
      MuonUT     & $\mu^-$ & $0.937 \pm 0.027$ \\
      \hline
      Combined   & $\mu^\pm$ & $0.939 \pm 0.007$ \\
      MuonUT     & $\mu^\pm$ & $0.944 \pm 0.017$ \\
      \hline
    \end{tabular}
  \end{center}
\end{frame}

\begin{frame}{Summary and next steps}
  \vspace{0.0cm}
  \begin{itemize}
    \setlength\itemsep{0.7em}
    \item{Matching to the wrong long track occurs more often with MuonUT method, potentially leading to biased tracking efficiencies}
    \item{After correcting for this using MC truth information, discrepancy seems to be gone, but uncertainties are currently large}
    \item{AP with more MC is currently running}
    \item{Next steps:}
    \begin{enumerate}
      \item{With more statistics, study effect in kinematic bins}
      \item{Figure out if this effect cancels in the data/MC ratio?}
    \end{enumerate}
  \end{itemize}
  \vspace{0.3cm}
  \begin{center}
    \Huge Thanks for listening!
  \end{center}
\end{frame}

\end{document}
