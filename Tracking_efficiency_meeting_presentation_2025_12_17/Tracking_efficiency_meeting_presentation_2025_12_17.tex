%%%%%%%%%%%%%%%%%%%%%%%%%%%%%%%%%%%%%%%%%%%%%%%%%%%%%%%%%%%%%%%%%%%%%%
% Overleaf (WriteLaTeX) Example: Molecular Chemistry Presentation
%
% Source: http://www.overleaf.com
%
% In these slides we show how Overleaf can be used with standard
% chemistry packages to easily create professional presentations.
%
% Feel free to distribute this example, but please keep the referral
% to overleaf.com
%
%%%%%%%%%%%%%%%%%%%%%%%%%%%%%%%%%%%%%%%%%%%%%%%%%%%%%%%%%%%%%%%%%%%%%%

\documentclass[xcolor={dvipsnames}]{beamer}

\mode<presentation>
{
  \usetheme{Madrid}       % or try default, Darmstadt, Warsaw, ...
  \usecolortheme{default} % or try albatross, beaver, crane, ...
  \usefonttheme{default}    % or try default, structurebold, ...
  \setbeamertemplate{navigation symbols}{}
  \setbeamertemplate{caption}[numbered]
}

\usepackage[english]{babel}
\usepackage[utf8x]{inputenc}
\usepackage{graphicx}
\usepackage{hyperref}
  \hypersetup{colorlinks=true}
  \hypersetup{urlcolor=blue}
  \hypersetup{linkcolor = .}
\usepackage{xcolor}
\usepackage{siunitx}
  \sisetup{separate-uncertainty = true}
\DeclareSIUnit\barn{b}
\usepackage{physics}
\usepackage[font=small,labelfont=bf]{caption}
\usepackage{subcaption}
\usepackage[en-GB]{datetime2}
\usepackage{overpic}
\usepackage{feynmp}
\DeclareGraphicsRule{*}{mps}{*}{}
\usepackage{scalerel}
\newcommand{\mylbrace}[2]{\vspace{#2pt}\hspace{6pt}\scaleleftright[\dimexpr5pt+#1\dimexpr0.06pt]{\lbrace}{\rule[\dimexpr2pt-#1\dimexpr0.5pt]{-4pt}{#1pt}}{.}}
\newcommand{\myrbrace}[2]{\vspace{#2pt}\scaleleftright[\dimexpr5pt+#1\dimexpr0.06pt]{.}{\rule[\dimexpr2pt-#1\dimexpr0.5pt]{-4pt}{#1pt}}{\rbrace}\hspace{6pt}}

% Trim in percent
\usepackage{adjustbox}

% No "Figure" prefix
\setbeamertemplate{caption}{\raggedright\insertcaption\par}

% Nice decay amplitude diagrams
\usepackage{amsmath,amssymb,tikz-cd}

% Strike out text
\usepackage[normalem]{ulem}

% For figures with text overlay
\usepackage{overpic}

% Arrows
\usepackage{tikz}
\newcommand{\tikzmark}[1]{\tikz[remember picture] \node[coordinate] (#1) {#1};}

% Colourbox with line breaks
\newcommand{\cbox}[2][lime!20]{%
  \colorbox{#1}{\parbox{\dimexpr\linewidth-2\fboxsep}{\strut #2\strut}}%
}

% Vector arrows
\usepackage[pdftex]{pict2e}

% Checkmark symbol
\def\checkmark{\tikz\fill[scale=0.4](0,.35) -- (.25,0) -- (1,.7) -- (.25,.15) -- cycle;}

% Here's where the presentation starts, with the info for the title slide
\title[Heidelberg tracking meeting]{Plan for finalising TrackCalib2}

\author[Martin Tat]{Martin Tat}
\institute[Heidelberg]{Heidelberg University}
\date{2nd December 2025}

\titlegraphic{\includegraphics[height = 2.3cm]{lhcb.jpg}\hspace{1.0cm}~%
              \includegraphics[height = 2.3cm]{HeidelbergLogo.pdf}}

\begin{document}

\begin{frame}
  \titlepage
\end{frame}

% These three lines create an automatically generated table of contents.
%\begin{frame}{Outline}
%  \tableofcontents
%\end{frame}

\begin{frame}{Introduction}
  \vspace{0.0cm}
  {\Large Current situation:}
  \vspace{0.2cm}
  \begin{itemize}
    \setlength\itemsep{0.8em}
    \item{Analysis productions produce tuples...}
    \item{... but there are two sets of analysis frameworks}
    \begin{enumerate}
      \item{Rowina's code}
      \item{TrackCalib2}
    \end{enumerate}
    \item{In principle both should do the same job $\implies$}
    \begin{itemize}
      \item{Use TrackCalib2 by default}
    \end{itemize}
  \end{itemize}
\end{frame}

\begin{frame}{Keeping TrackCalib2 up-to-date}
  \vspace{0.0cm}
  {\large Rowina, Maurice and I have gone through TrackCalib2 and Rowina's code to make sure everything is consistent}
  \vspace{0.2cm}
  \begin{enumerate}
    \setlength\itemsep{0.5em}
    \item{Selection criteria}
    \item{Matching criteria}
    \item{Fit range}
    \item{Latest AP with data/MC tuples}
    \item{Binning schemes}
    \begin{itemize}
      \item{Use Rowina's current 1D finer binning schemes in $p$ and $\eta$}
      \item{In 2D, start from the 1D x 1D binning and merge bins until each bin has a sufficient number of VeloMuon x Downstream candidates}
    \end{itemize}
  \end{enumerate}
  \begin{itemize}
    \item{Updating 1--4 in TrackCalib2 is work in progress}
    \item{Fancy 2D binning needs more thinking and code refactoring}
  \end{itemize}
\end{frame}

\begin{frame}{Developments in progess}
  \vspace{0.0cm}
  {\large TrackCalib2 needs more flexibility like Rowina's code:}
  \vspace{0.2cm}
  \begin{enumerate}
    \item{Tuple preparation is currently by far the biggest bottlekneck}
    \begin{itemize}
      \item{Aim: Perform offline selection of new tuples on-the-fly}
      \item{Current status: Tuples take hours to produce and each data block takes over 10 GB of space}
      \item{Bottlekneck: Uproot processes hundreds of files, but only one at the time, without any parallelisation}
      \item{Solution: Change to RDataFrame, which performs selections lazily and multithreaded, and only save necessary variables}
      \item{I am currently working on this}
    \end{itemize}
  \end{enumerate}
\end{frame}

\begin{frame}{Developments in progess}
  \vspace{0.0cm}
  {\large TrackCalib2 needs more flexibility like Rowina's code:}
  \vspace{0.2cm}
  \begin{enumerate}
    \setcounter{enumi}{1}
    \item{Fitting can take time when tweaking or doing studies}
    \begin{itemize}
      \item{Aim: Fit different 1D and 2D binning schemes, and allow for single-bin fits, potentially in parallel}
      \item{Current status: Same 1D and 2D binning is used, and every single 1D and 2D bin is fitted each time}
      \item{Solution: Need to rethink how to save fit results}
      \item{This is the next task}
      \item[]{}
      \item[]{}
    \end{itemize}
  \end{enumerate}
\end{frame}

\begin{frame}{What is the end goal?}
  \vspace{0.0cm}
  {\large Why am I doing this?}
  \vspace{0.2cm}
  \begin{itemize}
    \setlength\itemsep{0.8em}
    \item{Have a tool that can produce tracking efficiencies out-of-the-box}
    \item{Quickly produce correction tables when new data/MC is available}
    \item{Allows for more efficient debugging of current issues:}
    \begin{enumerate}
      \item{MuonUT vs Combined discrepancies}
      \item{Blocks 5/6 and 7/8 discrepancies}
    \end{enumerate}
  \end{itemize}
\end{frame}

\end{document}
