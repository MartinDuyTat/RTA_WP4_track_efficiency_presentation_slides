%%%%%%%%%%%%%%%%%%%%%%%%%%%%%%%%%%%%%%%%%%%%%%%%%%%%%%%%%%%%%%%%%%%%%%
% Overleaf (WriteLaTeX) Example: Molecular Chemistry Presentation
%
% Source: http://www.overleaf.com
%
% In these slides we show how Overleaf can be used with standard
% chemistry packages to easily create professional presentations.
%
% Feel free to distribute this example, but please keep the referral
% to overleaf.com
%
%%%%%%%%%%%%%%%%%%%%%%%%%%%%%%%%%%%%%%%%%%%%%%%%%%%%%%%%%%%%%%%%%%%%%%

\documentclass[xcolor={dvipsnames}]{beamer}

\mode<presentation>
{
  \usetheme{Madrid}       % or try default, Darmstadt, Warsaw, ...
  \usecolortheme{default} % or try albatross, beaver, crane, ...
  \usefonttheme{default}    % or try default, structurebold, ...
  \setbeamertemplate{navigation symbols}{}
  \setbeamertemplate{caption}[numbered]
}

\usepackage[english]{babel}
\usepackage[utf8x]{inputenc}
\usepackage{graphicx}
\usepackage{hyperref}
  \hypersetup{colorlinks=true}
  \hypersetup{urlcolor=blue}
  \hypersetup{linkcolor = .}
\usepackage{xcolor}
\usepackage{siunitx}
  \sisetup{separate-uncertainty = true}
\DeclareSIUnit\barn{b}
\usepackage{physics}
\usepackage[font=small,labelfont=bf]{caption}
\usepackage{subcaption}
\usepackage[en-GB]{datetime2}
\usepackage{overpic}
\usepackage{feynmp}
\DeclareGraphicsRule{*}{mps}{*}{}
\usepackage{scalerel}
\newcommand{\mylbrace}[2]{\vspace{#2pt}\hspace{6pt}\scaleleftright[\dimexpr5pt+#1\dimexpr0.06pt]{\lbrace}{\rule[\dimexpr2pt-#1\dimexpr0.5pt]{-4pt}{#1pt}}{.}}
\newcommand{\myrbrace}[2]{\vspace{#2pt}\scaleleftright[\dimexpr5pt+#1\dimexpr0.06pt]{.}{\rule[\dimexpr2pt-#1\dimexpr0.5pt]{-4pt}{#1pt}}{\rbrace}\hspace{6pt}}

% Trim in percent
\usepackage{adjustbox}

% No "Figure" prefix
\setbeamertemplate{caption}{\raggedright\insertcaption\par}

% Nice decay amplitude diagrams
\usepackage{amsmath,amssymb,tikz-cd}

% Strike out text
\usepackage[normalem]{ulem}

% For figures with text overlay
\usepackage{overpic}

% Arrows
\usepackage{tikz}
\newcommand{\tikzmark}[1]{\tikz[remember picture] \node[coordinate] (#1) {#1};}

% Colourbox with line breaks
\newcommand{\cbox}[2][lime!20]{%
  \colorbox{#1}{\parbox{\dimexpr\linewidth-2\fboxsep}{\strut #2\strut}}%
}

% Vector arrows
\usepackage[pdftex]{pict2e}

% Checkmark symbol
\def\checkmark{\tikz\fill[scale=0.4](0,.35) -- (.25,0) -- (1,.7) -- (.25,.15) -- cycle;}

% Here's where the presentation starts, with the info for the title slide
\title[Heidelberg tracking meeting]{Progress update on tracking efficiencies}

\author[Martin Tat]{Martin Tat}
\institute[Heidelberg]{Heidelberg University}
\date{23rd January 2026}

\titlegraphic{\includegraphics[height = 2.3cm]{lhcb.jpg}\hspace{1.0cm}~%
              \includegraphics[height = 2.3cm]{HeidelbergLogo.pdf}}

\begin{document}

\begin{frame}
  \titlepage
\end{frame}

% These three lines create an automatically generated table of contents.
%\begin{frame}{Outline}
%  \tableofcontents
%\end{frame}

\begin{frame}{Organisation of these meetings}
  \vspace{0.0cm}
  {\Large Current meeting time: Tuesdays 10:30}
  \vspace{0.2cm}
  \begin{itemize}
    \setlength\itemsep{0.8em}
    \item{Is this slot suitable for everyone? Or should we move it?}
    \begin{itemize}
      \item[-]{As RD RTA liason, I have biweekly RTA WP3 at 10:00 meetings}
    \end{itemize}
    \item{Note: As I'm not working full time on tracking efficiencies, it's likely I'll only organise these meetings biweekly instead of weekly}
    \item{I'll provide minor updates on MatterMost}
  \end{itemize}
\end{frame}

\begin{frame}{TrackCalib2 update: Environment}
  \vspace{0.0cm}
  \begin{itemize}
    \setlength\itemsep{1.0em}
    \item{Issue: The TrackCalib2 environment setup has caused headache}
    \begin{itemize}
      \item{Conda environment takes several hours to set up and occupies significant disk space}
      \item{Personally, the \texttt{mamba} instructions never worked for me on lxplus}
      \item{Different machines seem to throw different warnings/errors}
    \end{itemize}
    \item{Solution: Set up an \texttt{lb-conda} environment}
    \begin{itemize}
      \item{MR has been approved and merged: \href{https://gitlab.cern.ch/lhcb-core/conda-environments/-/merge_requests/247}{!247}}
      \item{I tested on lxplus and it worked out-of-the-box}
      \item{Anyone can clone TrackCalib2 and have it running in a few seconds}
    \end{itemize}
  \end{itemize}
\end{frame}

\begin{frame}{TrackCalib2 update: Environment}
  \vspace{0.0cm}
  \small
  {\large Clone TrackCalib2 repository:} \\
  \texttt{git clone ssh://git@gitlab.cern.ch:7999/lhcb-rta/trackcalib2.git} \\
  \texttt{git submodule init \&\& git submodule update} \\
  \vspace{1.0cm}
  {\large Load environment: $\leftarrow$ now takes seconds instead of hours} \\
  \texttt{lb-conda trackcalib2} \\
  \vspace{1.0cm}
  {\large Run TrackCalib2:} \\
  \texttt{python trackcalib.py prepare -year 2024\_Block1} \\
  \texttt{python trackcalib.py fit -year 2024\_Block1}
\end{frame}

\begin{frame}{Missing DD4HEP binds}
  \vspace{0.0cm}
  \begin{figure}[htb]
    \centering
    \includegraphics[width=1.0\textwidth,page=5]{DD4HEP_binds_screenshot.png}
  \end{figure}
  \begin{itemize}
    \setlength\itemsep{1.0em}
    \item{Do I understand correctly that multiple scattering and energy loss corrections are disabled for data, but not MC? Why...?}
    \item{I ran a quick comparison between MC with and without these corrections, and differences are huge for VeloMuon and MuonUT}
  \end{itemize}
\end{frame}

\begin{frame}{Missing DD4HEP binds}
  \vspace{0.0cm}
  \begin{figure}[htb]
    \centering
    \begin{subfigure}{0.5\textwidth}
      \centering
      \includegraphics[width=1.0\textwidth]{Plots/dd4hep_comparison_VeloMuon_mup.pdf}
    \end{subfigure}%
    \begin{subfigure}{0.5\textwidth}
      \centering
      \includegraphics[width=1.0\textwidth]{Plots/dd4hep_comparison_VeloMuon_mum.pdf}
    \end{subfigure}
  \end{figure}
  \begin{itemize}
    \setlength\itemsep{1.0em}
    \item{Fewer candidates with DD4HEP binds for VeloMuon}
  \end{itemize}
\end{frame}

\begin{frame}{Missing DD4HEP binds}
  \vspace{0.0cm}
  \begin{figure}[htb]
    \centering
    \begin{subfigure}{0.5\textwidth}
      \centering
      \includegraphics[width=1.0\textwidth]{Plots/dd4hep_comparison_MuonUT_mup.pdf}
    \end{subfigure}%
    \begin{subfigure}{0.5\textwidth}
      \centering
      \includegraphics[width=1.0\textwidth]{Plots/dd4hep_comparison_MuonUT_mum.pdf}
    \end{subfigure}
  \end{figure}
  \begin{itemize}
    \setlength\itemsep{1.0em}
    \item{Difference is much larger for MuonUT!}
  \end{itemize}
\end{frame}

\begin{frame}{Missing DD4HEP binds}
  \vspace{0.0cm}
  \begin{figure}[htb]
    \centering
    \begin{subfigure}{0.5\textwidth}
      \centering
      \includegraphics[width=0.7\textwidth]{Plots/dd4hep_comparison_VeloMuon_mup_Jpsi_BPVFDCHI2_log.pdf}
    \end{subfigure}%
    \begin{subfigure}{0.5\textwidth}
      \centering
      \includegraphics[width=0.7\textwidth]{Plots/dd4hep_comparison_VeloMuon_mup_muprobe_TRACK_CHI2NDOF.pdf}
    \end{subfigure}
    \begin{subfigure}{0.5\textwidth}
      \centering
      \includegraphics[width=0.7\textwidth]{Plots/dd4hep_comparison_VeloMuon_mup_muprobe_IPCHI2_OWNPV_log.pdf}
    \end{subfigure}%
    \begin{subfigure}{0.5\textwidth}
      \centering
      \includegraphics[width=0.7\textwidth]{Plots/dd4hep_comparison_VeloMuon_mup_muprobe_PT.pdf}
    \end{subfigure}
  \end{figure}
  \begin{itemize}
    \setlength\itemsep{0.0em}
    \item{It's mostly VeloMuon probe muon distributions that are different...}
    \item{... but I don't see much difference in MuonUT distributions}
  \end{itemize}
\end{frame}

\begin{frame}{Sprucing decision filter}
  \vspace{0.0cm}
  \begin{itemize}
    \setlength\itemsep{1.0em}
    \item{Reminder: Michel discovered a prescale in blocks 7/8}
    \item{Events failing the prescale but triggered by other lines are kept when rerunning reconstruction}
    \item{Solution: Apply sprucing decision filter directly in Moore}
    \item{Rowina discovered that applying the filter in blocks 5/6 also changed the tracking efficiencies, and I confirmed this in TrackCalib2 as well...}
    \item{... and for the number of $J/\psi$ candidates, I see a $10\%$ discrepancy (!!!) in block 5/6}
  \end{itemize}
\end{frame}

\begin{frame}{Sprucing decision filter}
  \vspace{0.0cm}
  \begin{itemize}
    \setlength\itemsep{1.0em}
    \item{Check impact of DD4HEP binds using block 6 data}
    \item{The numbers are surprisingly consistent with Moore v55r12p3:}
  \end{itemize}
  \begin{center}
    \begin{tabular}{llrrc}
      \hline
      Spruce   decision & DD4HEP binds & VeloMuon  & Downstream & MuonUT   \\
      \hline
      No                & No           &  197/183  &  719/706   &  121/45  \\
      Yes               & No           &  180/169  &  719/706   &  115/39  \\
      No                & Yes          &  195/178  &  719/706   &  131/46  \\
      Yes               & Yes          &  195/178  &  719/706   &  131/46  \\
      \hline
    \end{tabular}
  \end{center}
\end{frame}

\begin{frame}{Sprucing decision filter}
  \vspace{0.0cm}
  \begin{itemize}
    \setlength\itemsep{1.0em}
    \item{We need to run Moore v55r12p8 due to bug fixes}
    \item{Here there are differences, but they are very small:}
  \end{itemize}
  \begin{center}
    \begin{tabular}{llrrc}
      \hline
      Spruce decision   & DD4HEP binds & VeloMuon  & Downstream & MuonUT   \\
      \hline
      No                & No           &  193/182  &  665/665   &  114/43  \\
      Yes               & No           &  176/168  &  661/660   &  108/37  \\
      No                & Yes          &  195/172  &  665/665   &  123/44  \\
      Yes               & Yes          &  193/172  &  661/660   &  123/44  \\
      \hline
    \end{tabular}
  \end{center}
\end{frame}

\begin{frame}{Missing DD4HEP binds and sprucing decision filter}
  \vspace{0.0cm}
  \begin{itemize}
    \setlength\itemsep{1.0em}
    \item{Sprucing decision filter definitely needed to get correct efficiencies in blocks 7/8}
    \item{But clearly we don't get consistent numbers in blocks 5/6 unless DD4HEP binds are included}
    \item{Don't quite understand the DD4HEP binds yet...}
    \item{... but I will rerun all data APs with them}
  \end{itemize}
\end{frame}

\begin{frame}{TrackCalib2 development 1}
  \vspace{0.0cm}
  {\large TrackCalib development 1: Remove Run 1/2 code}
  \begin{itemize}
    \setlength\itemsep{1.0em}
    \item{Run 1/2 used different methods/variables/cuts/matching... huge amount of code unecessary for Run 3}
    \item{I suggest separate releases for Run 1/2 and Run 3}
    \begin{itemize}
      \item{Less code $\implies$ Easier to maintain}
    \end{itemize}
  \end{itemize}
\end{frame}

\begin{frame}{TrackCalib2 development 2}
  \vspace{0.0cm}
  {\large TrackCalib development 2: Change to RDataFrame}
  \begin{itemize}
    \setlength\itemsep{1.0em}
    \item{Uproot processes hundreds of files one by one}
    \item{Preparing samples can take hours}
    \begin{itemize}
      \item{Very difficult to perform studies that require reprocessing the samples}
    \end{itemize}
    \item{I know ROOT dependency was removed in the past, but for sample preparation RDataFrame is much more suitable in my opinion}
    \begin{itemize}
      \item{Multi-thread support}
      \item{Lazy evaluation $\implies$ Only need to loop once to apply all cuts and matching criteria}
    \end{itemize}
    \item{After this change, sample preparation only takes a few minutes!}
    \begin{itemize}
      \item{Allows me to prepare new samples and produce tracking efficiencies almost instantly once AP is ready}
    \end{itemize}
  \end{itemize}
\end{frame}

\begin{frame}{TrackCalib2 development 3}
  \vspace{0.0cm}
  {\large TrackCalib development 3: New data/MC samples}
  \begin{itemize}
    \setlength\itemsep{1.0em}
    \item{Sim10d $\to$ Sim10g}
    \item{Use overlap functors for matching}
    \item{All cuts and matching criteria updated (after discussion with Maurice and Rowina)}
    \item{Tracking efficiencies were compared before and after to understand differences}
    \begin{itemize}
      \item{Small differences understood to be sprucing decision filter and wrong/outdated matching criteria in TrackCalib2}
    \end{itemize}
    \item{Note: Plots I'm going to show don't have DD4HEP binds, which is expected to have a large impact}
  \end{itemize}
\end{frame}

\begin{frame}{TrackCalib2 development 3}
  \vspace{0.0cm}
  {\large TrackCalib development 3: New data/MC samples}
  \begin{figure}[htb]
    \centering
    \begin{subfigure}{0.5\textwidth}
      \centering
      \includegraphics[width=0.7\textwidth]{Plots/trackeff_MC_VeloMuon_ETA_Block1_old_new_comparison.pdf}
    \end{subfigure}%
    \begin{subfigure}{0.5\textwidth}
      \centering
      \includegraphics[width=0.7\textwidth]{Plots/trackeff_MC_VeloMuon_ETA_Block5_old_new_comparison.pdf}
    \end{subfigure}
    \begin{subfigure}{0.5\textwidth}
      \centering
      \includegraphics[width=0.7\textwidth]{Plots/trackeff_MC_VeloMuon_ETA_Block8_old_new_comparison.pdf}
    \end{subfigure}%
    \begin{subfigure}{0.5\textwidth}
      \centering
      \includegraphics[width=0.7\textwidth]{Plots/trackeff_MC_VeloMuon_ETA_new_samples.pdf}
    \end{subfigure}
  \end{figure}
  {MC VeloMuon track efficiencies in $\eta$}
\end{frame}

\begin{frame}{TrackCalib2 development 3}
  \vspace{0.0cm}
  {\large TrackCalib development 3: New data/MC samples}
  \begin{figure}[htb]
    \centering
    \begin{subfigure}{0.5\textwidth}
      \centering
      \includegraphics[width=0.7\textwidth]{Plots/trackeff_MC_VeloMuon_P_Block1_old_new_comparison.pdf}
    \end{subfigure}%
    \begin{subfigure}{0.5\textwidth}
      \centering
      \includegraphics[width=0.7\textwidth]{Plots/trackeff_MC_VeloMuon_P_Block5_old_new_comparison.pdf}
    \end{subfigure}
    \begin{subfigure}{0.5\textwidth}
      \centering
      \includegraphics[width=0.7\textwidth]{Plots/trackeff_MC_VeloMuon_P_Block8_old_new_comparison.pdf}
    \end{subfigure}%
    \begin{subfigure}{0.5\textwidth}
      \centering
      \includegraphics[width=0.7\textwidth]{Plots/trackeff_MC_VeloMuon_P_new_samples.pdf}
    \end{subfigure}
  \end{figure}
  {MC VeloMuon track efficiencies in $p$}
\end{frame}

\begin{frame}{TrackCalib2 development 3}
  \vspace{0.0cm}
  {\large TrackCalib development 3: New data/MC samples}
  \begin{figure}[htb]
    \centering
    \begin{subfigure}{0.5\textwidth}
      \centering
      \includegraphics[width=0.7\textwidth]{Plots/trackeff_MC_Downstream_ETA_Block1_old_new_comparison.pdf}
    \end{subfigure}%
    \begin{subfigure}{0.5\textwidth}
      \centering
      \includegraphics[width=0.7\textwidth]{Plots/trackeff_MC_Downstream_ETA_Block5_old_new_comparison.pdf}
    \end{subfigure}
    \begin{subfigure}{0.5\textwidth}
      \centering
      \includegraphics[width=0.7\textwidth]{Plots/trackeff_MC_Downstream_ETA_Block8_old_new_comparison.pdf}
    \end{subfigure}%
    \begin{subfigure}{0.5\textwidth}
      \centering
      \includegraphics[width=0.7\textwidth]{Plots/trackeff_MC_Downstream_ETA_new_samples.pdf}
    \end{subfigure}
  \end{figure}
  {MC Downstream track efficiencies in $\eta$}
\end{frame}

\begin{frame}{TrackCalib2 development 3}
  \vspace{0.0cm}
  {\large TrackCalib development 3: New data/MC samples}
  \begin{figure}[htb]
    \centering
    \begin{subfigure}{0.5\textwidth}
      \centering
      \includegraphics[width=0.7\textwidth]{Plots/trackeff_MC_Downstream_P_Block1_old_new_comparison.pdf}
    \end{subfigure}%
    \begin{subfigure}{0.5\textwidth}
      \centering
      \includegraphics[width=0.7\textwidth]{Plots/trackeff_MC_Downstream_P_Block5_old_new_comparison.pdf}
    \end{subfigure}
    \begin{subfigure}{0.5\textwidth}
      \centering
      \includegraphics[width=0.7\textwidth]{Plots/trackeff_MC_Downstream_P_Block8_old_new_comparison.pdf}
    \end{subfigure}%
    \begin{subfigure}{0.5\textwidth}
      \centering
      \includegraphics[width=0.7\textwidth]{Plots/trackeff_MC_Downstream_P_new_samples.pdf}
    \end{subfigure}
  \end{figure}
  {MC Downstream track efficiencies in $p$}
\end{frame}

\begin{frame}{TrackCalib2 development 3}
  \vspace{0.0cm}
  {\large TrackCalib development 3: New data/MC samples}
  \begin{figure}[htb]
    \centering
    \begin{subfigure}{0.5\textwidth}
      \centering
      \includegraphics[width=0.7\textwidth]{Plots/trackeff_MC_MuonUT_ETA_Block1_old_new_comparison.pdf}
    \end{subfigure}%
    \begin{subfigure}{0.5\textwidth}
      \centering
      \includegraphics[width=0.7\textwidth]{Plots/trackeff_MC_MuonUT_ETA_Block5_old_new_comparison.pdf}
    \end{subfigure}
    \begin{subfigure}{0.5\textwidth}
      \centering
      \includegraphics[width=0.7\textwidth]{Plots/trackeff_MC_MuonUT_ETA_Block8_old_new_comparison.pdf}
    \end{subfigure}%
    \begin{subfigure}{0.5\textwidth}
      \centering
      \includegraphics[width=0.7\textwidth]{Plots/trackeff_MC_MuonUT_ETA_new_samples.pdf}
    \end{subfigure}
  \end{figure}
  {MC MuonUT track efficiencies in $\eta$}
\end{frame}

\begin{frame}{TrackCalib2 development 3}
  \vspace{0.0cm}
  {\large TrackCalib development 3: New data/MC samples}
  \begin{figure}[htb]
    \centering
    \begin{subfigure}{0.5\textwidth}
      \centering
      \includegraphics[width=0.7\textwidth]{Plots/trackeff_MC_MuonUT_P_Block1_old_new_comparison.pdf}
    \end{subfigure}%
    \begin{subfigure}{0.5\textwidth}
      \centering
      \includegraphics[width=0.7\textwidth]{Plots/trackeff_MC_MuonUT_P_Block5_old_new_comparison.pdf}
    \end{subfigure}
    \begin{subfigure}{0.5\textwidth}
      \centering
      \includegraphics[width=0.7\textwidth]{Plots/trackeff_MC_MuonUT_P_Block8_old_new_comparison.pdf}
    \end{subfigure}%
    \begin{subfigure}{0.5\textwidth}
      \centering
      \includegraphics[width=0.7\textwidth]{Plots/trackeff_MC_MuonUT_P_new_samples.pdf}
    \end{subfigure}
  \end{figure}
  {MC MuonUT track efficiencies in $p$}
\end{frame}

\begin{frame}{TrackCalib2 development 3}
  \vspace{0.0cm}
  {\large TrackCalib development 3: New data/MC samples}
  \begin{figure}[htb]
    \centering
    \begin{subfigure}{0.5\textwidth}
      \centering
      \includegraphics[width=0.7\textwidth]{Plots/trackeff_Data_VeloMuon_ETA_Block1_old_new_comparison.pdf}
    \end{subfigure}%
    \begin{subfigure}{0.5\textwidth}
      \centering
      \includegraphics[width=0.7\textwidth]{Plots/trackeff_Data_VeloMuon_ETA_Block5_old_new_comparison.pdf}
    \end{subfigure}
    \begin{subfigure}{0.5\textwidth}
      \centering
      \includegraphics[width=0.7\textwidth]{Plots/trackeff_Data_VeloMuon_ETA_Block8_old_new_comparison.pdf}
    \end{subfigure}%
    \begin{subfigure}{0.5\textwidth}
      \centering
      \includegraphics[width=0.7\textwidth]{Plots/trackeff_Data_VeloMuon_ETA_new_samples.pdf}
    \end{subfigure}
  \end{figure}
  {Data VeloMuon track efficiencies in $\eta$}
\end{frame}

\begin{frame}{TrackCalib2 development 3}
  \vspace{0.0cm}
  {\large TrackCalib development 3: New data/MC samples}
  \begin{figure}[htb]
    \centering
    \begin{subfigure}{0.5\textwidth}
      \centering
      \includegraphics[width=0.7\textwidth]{Plots/trackeff_Data_VeloMuon_P_Block1_old_new_comparison.pdf}
    \end{subfigure}%
    \begin{subfigure}{0.5\textwidth}
      \centering
      \includegraphics[width=0.7\textwidth]{Plots/trackeff_Data_VeloMuon_P_Block5_old_new_comparison.pdf}
    \end{subfigure}
    \begin{subfigure}{0.5\textwidth}
      \centering
      \includegraphics[width=0.7\textwidth]{Plots/trackeff_Data_VeloMuon_P_Block8_old_new_comparison.pdf}
    \end{subfigure}%
    \begin{subfigure}{0.5\textwidth}
      \centering
      \includegraphics[width=0.7\textwidth]{Plots/trackeff_Data_VeloMuon_P_new_samples.pdf}
    \end{subfigure}
  \end{figure}
  {Data VeloMuon track efficiencies in $p$}
\end{frame}

\begin{frame}{TrackCalib2 development 3}
  \vspace{0.0cm}
  {\large TrackCalib development 3: New data/MC samples}
  \begin{figure}[htb]
    \centering
    \begin{subfigure}{0.5\textwidth}
      \centering
      \includegraphics[width=0.7\textwidth]{Plots/trackeff_Data_Downstream_ETA_Block1_old_new_comparison.pdf}
    \end{subfigure}%
    \begin{subfigure}{0.5\textwidth}
      \centering
      \includegraphics[width=0.7\textwidth]{Plots/trackeff_Data_Downstream_ETA_Block5_old_new_comparison.pdf}
    \end{subfigure}
    \begin{subfigure}{0.5\textwidth}
      \centering
      \includegraphics[width=0.7\textwidth]{Plots/trackeff_Data_Downstream_ETA_Block8_old_new_comparison.pdf}
    \end{subfigure}%
    \begin{subfigure}{0.5\textwidth}
      \centering
      \includegraphics[width=0.7\textwidth]{Plots/trackeff_Data_Downstream_ETA_new_samples.pdf}
    \end{subfigure}
  \end{figure}
  {Data Downstream track efficiencies in $\eta$}
\end{frame}

\begin{frame}{TrackCalib2 development 3}
  \vspace{0.0cm}
  {\large TrackCalib development 3: New data/MC samples}
  \begin{figure}[htb]
    \centering
    \begin{subfigure}{0.5\textwidth}
      \centering
      \includegraphics[width=0.7\textwidth]{Plots/trackeff_Data_Downstream_P_Block1_old_new_comparison.pdf}
    \end{subfigure}%
    \begin{subfigure}{0.5\textwidth}
      \centering
      \includegraphics[width=0.7\textwidth]{Plots/trackeff_Data_Downstream_P_Block5_old_new_comparison.pdf}
    \end{subfigure}
    \begin{subfigure}{0.5\textwidth}
      \centering
      \includegraphics[width=0.7\textwidth]{Plots/trackeff_Data_Downstream_P_Block8_old_new_comparison.pdf}
    \end{subfigure}%
    \begin{subfigure}{0.5\textwidth}
      \centering
      \includegraphics[width=0.7\textwidth]{Plots/trackeff_Data_Downstream_P_new_samples.pdf}
    \end{subfigure}
  \end{figure}
  {Data Downstream track efficiencies in $p$}
\end{frame}

\begin{frame}{TrackCalib2 development 3}
  \vspace{0.0cm}
  {\large TrackCalib development 3: New data/MC samples}
  \begin{figure}[htb]
    \centering
    \begin{subfigure}{0.5\textwidth}
      \centering
      \includegraphics[width=0.7\textwidth]{Plots/trackeff_Data_MuonUT_ETA_Block1_old_new_comparison.pdf}
    \end{subfigure}%
    \begin{subfigure}{0.5\textwidth}
      \centering
      \includegraphics[width=0.7\textwidth]{Plots/trackeff_Data_MuonUT_ETA_Block5_old_new_comparison.pdf}
    \end{subfigure}
    \begin{subfigure}{0.5\textwidth}
      \centering
      \includegraphics[width=0.7\textwidth]{Plots/trackeff_Data_MuonUT_ETA_Block8_old_new_comparison.pdf}
    \end{subfigure}%
    \begin{subfigure}{0.5\textwidth}
      \centering
      \includegraphics[width=0.7\textwidth]{Plots/trackeff_Data_MuonUT_ETA_new_samples.pdf}
    \end{subfigure}
  \end{figure}
  {Data MuonUT track efficiencies in $\eta$}
\end{frame}

\begin{frame}{TrackCalib2 development 3}
  \vspace{0.0cm}
  {\large TrackCalib development 3: New data/MC samples}
  \begin{figure}[htb]
    \centering
    \begin{subfigure}{0.5\textwidth}
      \centering
      \includegraphics[width=0.7\textwidth]{Plots/trackeff_Data_MuonUT_P_Block1_old_new_comparison.pdf}
    \end{subfigure}%
    \begin{subfigure}{0.5\textwidth}
      \centering
      \includegraphics[width=0.7\textwidth]{Plots/trackeff_Data_MuonUT_P_Block5_old_new_comparison.pdf}
    \end{subfigure}
    \begin{subfigure}{0.5\textwidth}
      \centering
      \includegraphics[width=0.7\textwidth]{Plots/trackeff_Data_MuonUT_P_Block8_old_new_comparison.pdf}
    \end{subfigure}%
    \begin{subfigure}{0.5\textwidth}
      \centering
      \includegraphics[width=0.7\textwidth]{Plots/trackeff_Data_MuonUT_P_new_samples.pdf}
    \end{subfigure}
  \end{figure}
  {Data MuonUT track efficiencies in $p$}
\end{frame}

\begin{frame}{TrackCalib2 development 3}
  \vspace{0.0cm}
  {\large TrackCalib development 3: New data/MC samples}
  \begin{figure}[htb]
    \centering
    \begin{subfigure}{0.5\textwidth}
      \centering
      \includegraphics[width=0.7\textwidth]{Plots/trackeff_ratio_ETA_1_new_samples.pdf}
    \end{subfigure}
    \begin{subfigure}{0.5\textwidth}
      \centering
      \includegraphics[width=0.7\textwidth]{Plots/trackeff_ratio_ETA_5_new_samples.pdf}
    \end{subfigure}%
    \begin{subfigure}{0.5\textwidth}
      \centering
      \includegraphics[width=0.7\textwidth]{Plots/trackeff_ratio_ETA_8_new_samples.pdf}
    \end{subfigure}
  \end{figure}
  {Data/MC ratio for Combined/MuonUT track efficiencies in $\eta$}
\end{frame}

\begin{frame}{TrackCalib2 development 3}
  \vspace{0.0cm}
  {\large TrackCalib development 3: New data/MC samples}
  \begin{figure}[htb]
    \centering
    \begin{subfigure}{0.5\textwidth}
      \centering
      \includegraphics[width=0.7\textwidth]{Plots/trackeff_ratio_P_1_new_samples.pdf}
    \end{subfigure}
    \begin{subfigure}{0.5\textwidth}
      \centering
      \includegraphics[width=0.7\textwidth]{Plots/trackeff_ratio_P_5_new_samples.pdf}
    \end{subfigure}%
    \begin{subfigure}{0.5\textwidth}
      \centering
      \includegraphics[width=0.7\textwidth]{Plots/trackeff_ratio_P_8_new_samples.pdf}
    \end{subfigure}
  \end{figure}
  {Data/MC ratio for Combined/MuonUT track efficiencies in $\eta$}
\end{frame}

\begin{frame}{Preparing data/MC samples}
  \vspace{0.0cm}
  \begin{enumerate}
    \setlength\itemsep{0.4em}
    \item{AP for blocks 1, 5, 6, 7, 8 of 2024 data}
    \begin{itemize}
      \item{AP ready: \href{https://gitlab.cern.ch/lhcb-datapkg/AnalysisProductions/-/merge_requests/4808}{!4808}}
      \item{Include sprucing decision and DD4HEP binds}
      \item{Apply tag muon TOS filter (removes $75\%$ of candidates)}
      \item{Remove tag muon overlap variables (removes $20\%$ of branches)}
    \end{itemize}
    \item{We should run over block 2 as well}
    \begin{itemize}
      \item{I requested MC yesterday}
    \end{itemize}
    \item{MC for block 1b}
    \begin{itemize}
      \item{MC request has been stuck for 1 month: \href{https://gitlab.cern.ch/lhcb-simulation/mc-requests/-/merge_requests/1640}{!1640}}
    \end{itemize}
    \item{2025 pre-TS data}
    \begin{itemize}
      \item{MC is ready, need to create AP over data/MC}
    \end{itemize}
    \item{2024 $pp$ reference run}
    \begin{itemize}
      \item{MC is ready, need to create AP over data/MC}
    \end{itemize}
    \item{Tidy up AP}
    \begin{itemize}
      \item{Perhaps remove older samples that are not used}
      \item{Track efficiency AP currently takes up 13 TB(!)}
    \end{itemize}
  \end{enumerate}
\end{frame}

\begin{frame}{Summary}
  \vspace{0.0cm}
  \begin{enumerate}
    \setlength\itemsep{1.0em}
    \item{Standard environment for TrackCalib2 works}
    \item{Briefly looked at impact of missing DD4HEP binds, impact seems to be significant but not understood yet}
    \item{Sprucing decision inconsistency in block 5/6 has been resolved, understood to be inconsistent DD4HEP binds when rerunning reconstruction}
    \item{Several updates to TrackCalib2:}
    \begin{itemize}
      \item{Run 1/2 stuff removed, we should create a separate release for Run 3}
      \item{RDataFrame for sample preparation is much faster}
      \item{New data/MC samples}
      \begin{itemize}
        \item[-]{Overlap functors for matching}
        \item[-]{Sprucing decision filter included}
      \end{itemize}
      \item{New tracking efficiencies are consistent with previous samples once accounting for new matching criteria and sprucing decision}
    \end{itemize}
  \end{enumerate}
\end{frame}

\begin{frame}{Next steps}
  \vspace{0.0cm}
  \begin{enumerate}
    \setlength\itemsep{1.5em}
    \item{Submit AP over 2024 data and rerun everything}
    \item{Look at 2025 pre-TS tracking efficiencies}
    \item{Add block 2 and $pp$ ref run}
    \item{Optimise 2D binning scheme and produce 2D data/MC ratio tables for all 2024 blocks and $\mu^\pm$, $\mu^+$, $\mu^-$}
    \item{Unfortunately I've had limited time to look into Combined/MuonUT discrepancies, but I'll try to do this in parallel}
  \end{enumerate}
\end{frame}

\end{document}
