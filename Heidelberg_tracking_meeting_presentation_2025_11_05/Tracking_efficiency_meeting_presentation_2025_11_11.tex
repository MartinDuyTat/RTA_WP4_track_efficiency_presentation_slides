%%%%%%%%%%%%%%%%%%%%%%%%%%%%%%%%%%%%%%%%%%%%%%%%%%%%%%%%%%%%%%%%%%%%%%
% Overleaf (WriteLaTeX) Example: Molecular Chemistry Presentation
%
% Source: http://www.overleaf.com
%
% In these slides we show how Overleaf can be used with standard
% chemistry packages to easily create professional presentations.
%
% Feel free to distribute this example, but please keep the referral
% to overleaf.com
%
%%%%%%%%%%%%%%%%%%%%%%%%%%%%%%%%%%%%%%%%%%%%%%%%%%%%%%%%%%%%%%%%%%%%%%

\documentclass[xcolor={dvipsnames}]{beamer}

\mode<presentation>
{
  \usetheme{Madrid}       % or try default, Darmstadt, Warsaw, ...
  \usecolortheme{default} % or try albatross, beaver, crane, ...
  \usefonttheme{default}    % or try default, structurebold, ...
  \setbeamertemplate{navigation symbols}{}
  \setbeamertemplate{caption}[numbered]
}

\usepackage[english]{babel}
\usepackage[utf8x]{inputenc}
\usepackage{graphicx}
\usepackage{hyperref}
  \hypersetup{colorlinks=true}
  \hypersetup{urlcolor=blue}
  \hypersetup{linkcolor = .}
\usepackage{xcolor}
\usepackage{siunitx}
  \sisetup{separate-uncertainty = true}
\DeclareSIUnit\barn{b}
\usepackage{physics}
\usepackage[font=small,labelfont=bf]{caption}
\usepackage{subcaption}
\usepackage[en-GB]{datetime2}
\usepackage{overpic}
\usepackage{feynmp}
\DeclareGraphicsRule{*}{mps}{*}{}
\usepackage{scalerel}
\newcommand{\mylbrace}[2]{\vspace{#2pt}\hspace{6pt}\scaleleftright[\dimexpr5pt+#1\dimexpr0.06pt]{\lbrace}{\rule[\dimexpr2pt-#1\dimexpr0.5pt]{-4pt}{#1pt}}{.}}
\newcommand{\myrbrace}[2]{\vspace{#2pt}\scaleleftright[\dimexpr5pt+#1\dimexpr0.06pt]{.}{\rule[\dimexpr2pt-#1\dimexpr0.5pt]{-4pt}{#1pt}}{\rbrace}\hspace{6pt}}

% Trim in percent
\usepackage{adjustbox}

% No "Figure" prefix
\setbeamertemplate{caption}{\raggedright\insertcaption\par}

% Nice decay amplitude diagrams
\usepackage{amsmath,amssymb,tikz-cd}

% Strike out text
\usepackage[normalem]{ulem}

% For figures with text overlay
\usepackage{overpic}

% Arrows
\usepackage{tikz}
\newcommand{\tikzmark}[1]{\tikz[remember picture] \node[coordinate] (#1) {#1};}

% Colourbox with line breaks
\newcommand{\cbox}[2][lime!20]{%
  \colorbox{#1}{\parbox{\dimexpr\linewidth-2\fboxsep}{\strut #2\strut}}%
}

% Vector arrows
\usepackage[pdftex]{pict2e}

% Checkmark symbol
\def\checkmark{\tikz\fill[scale=0.4](0,.35) -- (.25,0) -- (1,.7) -- (.25,.15) -- cycle;}

% Here's where the presentation starts, with the info for the title slide
\title[Tracking efficiency meeting]{Effects of muon alignment in MuonUT method and HLT1 trigger efficiencies}

\author[Martin Tat]{Martin Tat}
\institute[Heidelberg]{Heidelberg University}
\date{11th November 2025}

\titlegraphic{\includegraphics[height = 2.3cm]{lhcb.jpg}\hspace{1.0cm}~%
              \includegraphics[height = 2.3cm]{HeidelbergLogo.pdf}}

\begin{document}

\begin{frame}
  \titlepage
\end{frame}

% These three lines create an automatically generated table of contents.
%\begin{frame}{Outline}
%  \tableofcontents
%\end{frame}

\begin{frame}{The MuonUT method}
  \vspace{0.0cm}
  \begin{enumerate}
    \setlength\itemsep{1.0em}
    \item{Get hits from Muon system}
    \item{Reconstruct standalone muon track}
    \begin{itemize}
    \setlength\itemsep{0.4em}
      \item{Four muon hits (M2, M3, M4, M5)}
      \item{Fit straight line in YZ and XZ planes}
      \item{Calculate $p_x$ kick from knowledge of magnet centre $z_{\rm magnet}$, assuming track originated from the origin}
    \end{itemize}
    \item{Extrapolate track to UT and add UT hits}
  \end{enumerate}
\end{frame}

\begin{frame}{Charge asymmetry in MuonUT tuples}
  \vspace{0.0cm}
  \begin{center}
    {\Large What is the issue?}
  \end{center}
  \vspace{0.3cm}
  \begin{itemize}
    \setlength\itemsep{1.0em}
    \item{Huge difference in the number of $\mu^+$ and $\mu^-$ candidates for 2024}
    \begin{itemize}
      \item{Only in data, not MC}
    \end{itemize}
    \item{Behaviour swaps between magnet polarities}
    \item{What is the cause?}
    \begin{enumerate}
      \item{Fewer tracks reconstructed on the C-side, compared to A-side}
      \item{Kinematic distributions, such as $p_T$ and $J/\psi$ $\chi_{\rm vtx}^2$, are shifted $\implies$ Effectively tighter cuts in trigger selection}
    \end{enumerate}
  \end{itemize}
\end{frame}

\begin{frame}{Charge asymmetry in MuonUT tuples}
  \vspace{0.0cm}
  \begin{center}
    {\Large How large is the issue? A factor \underline{two}!}
  \end{center}
  \vspace{0.5cm}
  \begin{center}
    \begin{tabular}{llrrc}
      \hline
      Sample       & Magnet polarity & $\mu^+$   & $\mu^-$   & Ratio $+/-$ \\
      \hline
      2024 block 1 & Up              & $1126660$ & $2046110$ & 0.55 \\
      2024 block 5 & Up              & $2739920$ & $5832372$ & 0.47 \\
      2024 block 6 & Down            & $5036676$ & $2322011$ & 2.17 \\
      2024 block 7 & Down            & $2430038$ & $1155671$ & 2.10 \\
      2024 block 8 & Up              & $702585$  & $1443764$ & 0.49 \\
      \hline
    \end{tabular}
  \end{center}
\end{frame}

\begin{frame}{Charge asymmetry in MuonUT tuples}
  \vspace{0.0cm}
  \begin{center}
    {\Large What about 2025?}
  \end{center}
  \vspace{0.2cm}
  \begin{itemize}
    \setlength\itemsep{1.0em}
    \item{No asymmetry in $\mu^+$ and $\mu^-$ candidates in 2025 data}
    \item{Kinematic distributions look much more symmetric in 2025}
  \end{itemize}
  \vspace{0.2cm}
  \begin{center}
    {\large Main changes in 2025 data taking (by Michel):}
  \end{center}
  \vspace{0.0cm}
  \begin{itemize}
    \setlength\itemsep{1.0em}
    \item{Use muon clusters instead of muon hits}
    \item{Constrain $y = \SI{0(20)}{\milli\meter}$ at $z = 0$ in linear fit in the YZ plane}
  \end{itemize}
  \vspace{0.2cm}
  \begin{center}
    {\large Additionally: Muon alignment updated in September 2025 (Sprucing25c3) (see \href{https://indico.cern.ch/event/1576157/\#7-update-muon-alignment}{here})}
  \end{center}
\end{frame}

\begin{frame}{Muon alignment}
  \vspace{0.0cm}
  \begin{center}
    {\Large My working assumption for the last few months: \\
      Muon system misalignment in $y$}
  \end{center}
  \vspace{0.5cm}
  \begin{itemize}
    \setlength\itemsep{1.0em}
    \item{Mis-aligned Muon system could bias the extrapolation to the UT}
    \item{UT hits might be be correctly added, or track quality might be worse}
    \item{Effect not seen in VeloMuon or downstream because hits from tracking detectors place stronger constraints on particle trajectory}
    \item{$y$-constraint added by Michel counteracts misalignment in 2025 data}
    \item{How to prove this hypothesis?}
  \end{itemize}
\end{frame}

\begin{frame}{Study of 2025 data}
  \vspace{0.0cm}
  \begin{center}
    {\Large Strategy for analysing 2025 data:}
  \end{center}
  \vspace{0.5cm}
  \begin{enumerate}
    \setlength\itemsep{1.0em}
    \item{Tuple VeloMuon events that also passed MuonUT trigger line}
    \begin{itemize}
      \item{Unbiased sample of muons to study alignment with}
    \end{itemize}
    \item{For the same events, create new tuple with muon tracks}
    \begin{itemize}
      \item{Rerun standalone muon track reconstruction without $y$-constraint}
    \end{itemize}
    \item{Match muon tracks to VeloMuon probe tracks using LHCbIDs}
    \begin{itemize}
      \item{Small issue: A small number of events with multiple muon track candidates with exactly the same LHCbIDs...?}
      \item{For now keep these, but I'm really scratching my head over this}
    \end{itemize}
    \item{Study $y$-position of muon tracks, extrapolated back to the origin}
  \end{enumerate}
\end{frame}

\begin{frame}{Sprucing25c3 MagUp alignment}
  \vspace{0.0cm}
  \begin{figure}[htb]
    \centering
    \begin{subfigure}{0.5\textwidth}
      \centering
      \includegraphics[width=1.0\textwidth]{Plots/MuonUT_extrapolated_y_mup_Sprucing25c3.pdf}
    \end{subfigure}%
    \begin{subfigure}{0.5\textwidth}
      \centering
      \includegraphics[width=1.0\textwidth]{Plots/MuonUT_extrapolated_y_mum_Sprucing25c3.pdf}
    \end{subfigure}
  \end{figure}
  \begin{itemize}
    \item{$\mu^+$ $\mu^-$ mostly hit the C-side (A-side) due to magnetic field}
    \item{Minor residual mis-alignment, but this is probably very close to the position resolution of the Muon system}
  \end{itemize}
\end{frame}

\begin{frame}{Sprucing25c1 MagUp alignment}
  \vspace{0.0cm}
  \begin{figure}[htb]
    \centering
    \begin{subfigure}{0.5\textwidth}
      \centering
      \includegraphics[width=1.0\textwidth]{Plots/MuonUT_extrapolated_y_mup_Sprucing25c1.pdf}
    \end{subfigure}%
    \begin{subfigure}{0.5\textwidth}
      \centering
      \includegraphics[width=1.0\textwidth]{Plots/MuonUT_extrapolated_y_mum_Sprucing25c1.pdf}
    \end{subfigure}
  \end{figure}
  \begin{itemize}
    \item{Huge ($\sim\SI{25}{\milli\meter}$) mis-alignment on the C-side}
    \item{Have checked with Chenxu Yu, the only change in Sprucing25c3 was the muon alignment}
  \end{itemize}
\end{frame}

\begin{frame}{Sprucing25c1 MagUp alignment}
  \vspace{0.0cm}
  \begin{figure}[htb]
    \centering
    \begin{subfigure}{0.5\textwidth}
      \centering
      \includegraphics[width=1.0\textwidth]{Plots/MuonUT_extrapolated_y_mup_Sprucing25c1_new_alignment.pdf}
    \end{subfigure}%
    \begin{subfigure}{0.5\textwidth}
      \centering
      \includegraphics[width=1.0\textwidth]{Plots/MuonUT_extrapolated_y_mum_Sprucing25c1_new_alignment.pdf}
    \end{subfigure}
  \end{figure}
  \begin{itemize}
    \item{Reconstructing Sprucing25c1 with newest muon alignment: No bias!}
    \item{However, I don't fully understand how a $\SI{5}{\milli\meter}$ misalignment in M3 can cause a $\SI{25}{\milli\meter}$ bias in $y$ at the origin}
  \end{itemize}
\end{frame}

\begin{frame}{Effect on trigger selection}
  \vspace{0.0cm}
  \begin{center}
    {\Large To get an unbiased quantification of the effect on the trigger selection:}
  \end{center}
  \vspace{0.5cm}
  \begin{itemize}
    \setlength\itemsep{1.0em}
    \item{Re-run trigger line selection in Moore in Sprucing25c1 data}
    \begin{enumerate}
      \item{With nominal settings}
      \item{Without $y$-constraint}
      \item{Without $y$-constraint and with new alignment}
    \end{enumerate}
    \item{2025 data does not have the charge asymmetry, so this should be unbiased evidence for my hypothesis}
  \end{itemize}
\end{frame}

\begin{frame}{Effect on trigger selection}
  \vspace{0.0cm}
  \begin{center}
    {\Large Retention when running MuonUT trigger line in Moore over $50\times 10^3$ events from Sprucing25c1}
  \end{center}
  \vspace{0.5cm}
  \begin{center}
    \begin{tabular}{llrrc}
      \hline
      $y$-constraint & Alignment & $\mu^+$ & $\mu^-$ & Ratio $+/-$ \\
      \hline
      Yes            & Old       & $388$   & $416$   & 0.93 \\
      No             & Old       &  $80$   & $199$   & 0.40 \\
      No             & New       & $184$   & $208$   & 0.88 \\
      \hline
    \end{tabular}
  \end{center}
\end{frame}

\begin{frame}{Summary on MuonUT charge asymmetry}
  \vspace{0.0cm}
  \begin{itemize}
    \setlength\itemsep{0.7em}
    \item{Studied impact of muon alignment on the MuonUT method by rerunning reconstruction on 2024 data without $y$-constraint}
    \item{Muon standalone tracks have a large mis-alignment on the C-side before September 2025}
    \item{Confirmed this by running Moore with/without alignment}
    \item{Next steps:}
    \begin{enumerate}
      \item{Decide whether or not this affect matched and failed samples identically}
    \end{enumerate}
  \end{itemize}
\end{frame}

\begin{frame}{HLT1 trigger efficiencies}
  \vspace{0.0cm}
  \begin{itemize}
    \setlength\itemsep{1.0em}
    \item{The most urgent issue currently is the significantly larger SciFi (VeloMuon) efficiencies for blocks 7/8, compared to blocks 5/6}
    \begin{itemize}
      \item{Seen in fits performed by Rowina}
      \item{Reproduced in TrackCalib2 by Rowina}
    \end{itemize}
    \item{I have tried to understand differences in the data samples}
    \begin{itemize}
      \item{For now look at positive muons in blocks 5 and 8}
      \item{Main difference: $\mu = 4.4$ vs $\mu = 5.3$}
      \item{HLT1 trigger thresholds have also changed}
    \end{itemize}
  \end{itemize}
\end{frame}

\begin{frame}{HLT1 trigger efficiencies}
  \vspace{0.0cm}
  \begin{itemize}
    \setlength\itemsep{1.0em}
    \item{Integrated luminosity:}
    \begin{itemize}
      \item{Block 5: $\SI{1.16}{\per\femto\barn}$}
      \item{Block 8: $\SI{0.44}{\per\femto\barn}$}
    \end{itemize}
    \item{Candidates per luminosity:}
    \begin{itemize}
      \item{Block 5: $13.3\times10^6$ per $\si{\per\femto\barn}$}
      \item{Block 8: $9.7\times10^6$ per $\si{\per\femto\barn}$}
    \end{itemize}
    \item{Signal candidates per luminosity (by sideband subtraction):}
    \begin{itemize}
      \item{Block 5: $4.2\times10^6$ per $\si{\per\femto\barn}$}
      \item{Block 8: $4.1\times10^6$ per $\si{\per\femto\barn}$}
    \end{itemize}
    \item{Background candidates per luminosity (from sidebands):}
    \begin{itemize}
      \item{Block 5: $3.1\times10^6$ per $\si{\per\femto\barn}$}
      \item{Block 8: $1.9\times10^6$ per $\si{\per\femto\barn}$}
    \end{itemize}
  \end{itemize}
\end{frame}

\begin{frame}{HLT1 trigger efficiencies}
  \vspace{0.0cm}
  \begin{itemize}
    \setlength\itemsep{1.0em}
    \item{To understand differences, check trends in TOS trigger efficiencies}
    \item{TOS selection:}
    \begin{itemize}
      \item{\texttt{Hlt1TrackMVA} or}
      \item{\texttt{Hlt1TrackMuonMVA}}
    \end{itemize}
    \item{TOS selection:}
    \begin{itemize}
      \item{\texttt{Hlt1TrackMVA} or}
      \item{\texttt{Hlt1TrackMuonMVA} or}
      \item{\texttt{Hlt1TwoTrackMVA}}
    \end{itemize}
    \item{Things I've looked at:}
    \begin{itemize}
      \item{Difference between blocks 5 and 8}
      \item{Difference between pass and fail samples}
      \item{Difference between data and MC}
    \end{itemize}
  \end{itemize}
\end{frame}

\begin{frame}{HLT1 TOS efficiency comparison between blocks}
  \vspace{0.0cm}
  \begin{figure}[htb]
    \centering
    \begin{subfigure}{0.5\textwidth}
      \centering
      \includegraphics[width=1.0\textwidth]{Plots/TOS_block_comparison_pass_ETA_bins_data.pdf}
    \end{subfigure}%
    \begin{subfigure}{0.5\textwidth}
      \centering
      \includegraphics[width=1.0\textwidth]{Plots/TOS_block_comparison_pass_ETA_bins_MC.pdf}
    \end{subfigure}
  \end{figure}
  \begin{center}
    Comparison between blocks 5 and 8, in bins of $\eta$, for matched candidates
  \end{center}
\end{frame}

\begin{frame}{HLT1 TOS efficiency comparison between blocks}
  \vspace{0.0cm}
  \begin{figure}[htb]
    \centering
    \begin{subfigure}{0.5\textwidth}
      \centering
      \includegraphics[width=1.0\textwidth]{Plots/TOS_block_comparison_pass_P_bins_data.pdf}
    \end{subfigure}%
    \begin{subfigure}{0.5\textwidth}
      \centering
      \includegraphics[width=1.0\textwidth]{Plots/TOS_block_comparison_pass_P_bins_MC.pdf}
    \end{subfigure}
  \end{figure}
  \begin{center}
    Comparison between blocks 5 and 8, in bins of $p$, for matched candidates
  \end{center}
\end{frame}

\begin{frame}{HLT1 TOS efficiency in matched and failed samples}
  \vspace{0.0cm}
  \begin{figure}[htb]
    \centering
    \begin{subfigure}{0.5\textwidth}
      \centering
      \includegraphics[width=1.0\textwidth]{Plots/TOS_pass_fail_comparison_Block5_ETA_data_bins.pdf}
    \end{subfigure}%
    \begin{subfigure}{0.5\textwidth}
      \centering
      \includegraphics[width=1.0\textwidth]{Plots/TOS_pass_fail_comparison_Block8_ETA_data_bins.pdf}
    \end{subfigure}
  \end{figure}
  \begin{center}
    Comparison between matched and failed candidates, in bins of $\eta$, for data
  \end{center}
\end{frame}

\begin{frame}{HLT1 TOS efficiency in matched and failed samples}
  \vspace{0.0cm}
  \begin{figure}[htb]
    \centering
    \begin{subfigure}{0.5\textwidth}
      \centering
      \includegraphics[width=1.0\textwidth]{Plots/TOS_pass_fail_comparison_Block5_P_data_bins.pdf}
    \end{subfigure}%
    \begin{subfigure}{0.5\textwidth}
      \centering
      \includegraphics[width=1.0\textwidth]{Plots/TOS_pass_fail_comparison_Block8_P_data_bins.pdf}
    \end{subfigure}
  \end{figure}
  \begin{center}
    Comparison between matched and failed candidates, in bins of $p$, for data
  \end{center}
\end{frame}

\begin{frame}{HLT1 TOS efficiency in matched and failed samples}
  \vspace{0.0cm}
  \begin{figure}[htb]
    \centering
    \begin{subfigure}{0.5\textwidth}
      \centering
      \includegraphics[width=1.0\textwidth]{Plots/TOS_pass_fail_comparison_Block5_ETA_MC_bins.pdf}
    \end{subfigure}%
    \begin{subfigure}{0.5\textwidth}
      \centering
      \includegraphics[width=1.0\textwidth]{Plots/TOS_pass_fail_comparison_Block5_ETA_MC_bins.pdf}
    \end{subfigure}
  \end{figure}
  \begin{center}
    Comparison between matched and failed candidates, in bins of $\eta$, for MC
  \end{center}
\end{frame}

\begin{frame}{HLT1 TOS efficiency in matched and failed samples}
  \vspace{0.0cm}
  \begin{figure}[htb]
    \centering
    \begin{subfigure}{0.5\textwidth}
      \centering
      \includegraphics[width=1.0\textwidth]{Plots/TOS_pass_fail_comparison_Block5_P_MC_bins.pdf}
    \end{subfigure}%
    \begin{subfigure}{0.5\textwidth}
      \centering
      \includegraphics[width=1.0\textwidth]{Plots/TOS_pass_fail_comparison_Block5_P_MC_bins.pdf}
    \end{subfigure}
  \end{figure}
  \begin{center}
    Comparison between matched and failed candidates, in bins of $p$, for MC
  \end{center}
\end{frame}

\begin{frame}{HLT1 TOS efficiency comparison between data and MC}
  \vspace{0.0cm}
  \begin{figure}[htb]
    \centering
    \begin{subfigure}{0.5\textwidth}
      \centering
      \includegraphics[width=1.0\textwidth]{Plots/TOS_data_MC_comparison_Block5_ETA_pass_bins.pdf}
    \end{subfigure}%
    \begin{subfigure}{0.5\textwidth}
      \centering
      \includegraphics[width=1.0\textwidth]{Plots/TOS_data_MC_comparison_Block8_ETA_pass_bins.pdf}
    \end{subfigure}
  \end{figure}
  \begin{center}
    Comparison between data and MC, in bins of $\eta$, for blocks 5 and 8
  \end{center}
\end{frame}

\begin{frame}{HLT1 TOS efficiency comparison between data and MC}
  \vspace{0.0cm}
  \begin{figure}[htb]
    \centering
    \begin{subfigure}{0.5\textwidth}
      \centering
      \includegraphics[width=1.0\textwidth]{Plots/TOS_data_MC_comparison_Block5_P_pass_bins.pdf}
    \end{subfigure}%
    \begin{subfigure}{0.5\textwidth}
      \centering
      \includegraphics[width=1.0\textwidth]{Plots/TOS_data_MC_comparison_Block8_P_pass_bins.pdf}
    \end{subfigure}
  \end{figure}
  \begin{center}
    Comparison between data and MC, in bins of $p$, for blocks 5 and 8
  \end{center}
\end{frame}

\begin{frame}{Summary on HLT1 TOS efficiencies}
  \vspace{0.0cm}
  \begin{itemize}
    \setlength\itemsep{0.7em}
    \item{Observe a significantly larger HLT1 TOS efficiency on matched sample in block 8}
    \begin{itemize}
      \item{Only in data}
      \item{Could be due to changes to HLT1}
      \item{Could this affect tracking efficiencies?}
    \end{itemize}
    \item{See some differences in TOS efficiencies between matched and failed samples}
    \begin{itemize}
      \item{Is this expected?}
    \end{itemize}
    \item{MC generally has a higher efficiency for matched sample, but statistics insufficient in failed sample to draw any conclusion}
    \item{Is there anything suspicious here that should be looked into in more detail?}
  \end{itemize}
\end{frame}

\end{document}
